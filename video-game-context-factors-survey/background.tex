%!TEX root = paper.tex
%%%%%%%%%%%%%%%%%%%%%%%%%%%%%%%%%%%%%%%%%%%%%%%%%%%%%%%%%%%%%%%%%%%%%%%%%%%%%%%%
\section{Background}
\label{sec:background}
This section introduces system influencing factors and service factors
of current gaming systems:
It presents the hardware and software platforms and ecosystems
for different types of gaming, introduces utility metrics for games,
and describes the data sources used for the evaluations in this paper.
Furthermore, the gamer survey is described.
All costs are from an European, specifically German, perspective. If a
product is not available in this region, the prices are converted using
the most recent currency exchange rates.

\subsection{Gaming Hardware}\label{sec:gaming-hardware}

\subsubsection{Gaming \acrshort{PC}s}
Hardware viable for \gls{PC} gaming starts at about \SI{500}[\EUR]{} but
has practically no upper limit for enthusiasts. The \gls{GPU}
is a cost driver, and it is essential for modern \gls{PC} gaming.
This poses a certain financial barrier for customers to start \gls{PC} gaming,
which is however compensated by an increased flexibility and longevity of
hardware.

\subsubsection{Video Game Consoles}
Dedicated consoles represent the classical approach to video gaming.
The price for (non-portable) consoles varies but usually lies between
\SI{300}[\EUR]{} and \SI{400}[\EUR]{} for the latest console
generation, i.e., \textit{Switch}, \textit{PlayStation 4}, and
\textit{Xbox One}.
While priced lower than many \glspl{PC}, console hardware is not built from user-serviceable components
and not upgradeable.

\subsubsection{Cloud Gaming Hardware}
One of the main claimed benefits of \gls{cg} is its low requirements
on end user hardware. Thus, a less powerful \gls{PC} may suffice to
use a \gls{cg} service as long as the network connection to the
service provides an adequate latency and bandwidth.
Sony's \psnow \gls{cg} service was originally available only for their
PlayStation 3 and 4 consoles, and a line of their TV sets. The latter
option has since been removed, and streaming to \glspl{PC} enabled
instead.
Companies also experiment with devices that focus primarily on streaming
\gls{cg} content. The most recent standalone game streaming device currently
in the market is NVIDIA's \textit{SHIELD} which starts at
\SI{230}[\EUR]{} and is bound to their \gls{cg} service, \gfnow.
Here, too, a stream-to-\gls{PC} option has become available recently
(as a US-only beta program).



\subsection{Gaming Ecosystems}
Below, currently active (cloud and non-cloud) gaming platforms are examined
with regards to pricing models, other requirements, and costs. The
information presented was collected between July 2015 and October 2017.

\subsubsection{\gls{PC} Gaming}
\label{sec:pcgaming}

The rise of easy-to-use digital distribution platforms and the
independent (``indie'') game scene reinvigorated \gls{PC} gaming just a few
years ago. Today, \gls{PC} gaming is dominated by digital marketplaces,
with \steam being the largest. The platform has about $10$ million
concurrent users at most times of the day. It periodically offers large,
often seasonal, sales of recent games at greatly reduced prices (rebates
of \SI{75}{\percent} for a year-old game are not uncommon). In addition, many
resellers offer digital codes for other platforms, often at much lower
prices.
Major releases on PC are usually priced between \SI{50}[\EUR]{}
and \SI{60}[\EUR]{}. However, due to the competition between the
vendors, the digital retail prices are significantly lower even at
launch, and also drop more quickly. Another recent trend are game
bundles, which especially prevail in the indie games scene, commonly
offered with a pay-what-you-want model. \textit{Humble
Bundle}\footnote{\url{https://www.humblebundle.com/}} is a prominent
example.

\subsubsection{Video Game Consoles}

New, major game releases are mostly priced at either
\SI{60}[\EUR]{} or \SI{70}[\EUR]{}. Once on the market, the
game prices decrease rather slowly. In recent years, retail stores have
been complemented with console-specific, proprietary digital
distribution services that also offer the latest game at the full price.
These official stores are usually exclusive vendors for digital game
codes where competitors are excluded.
Subscription fees often apply for the multiplayer mode of games, e.g.,
\textit{PlayStation Plus} or \textit{Xbox Live Gold} with annual prices
of \SI{50}[\EUR]{} and \SI{60}[\EUR]{}, respectively. These
services also include access to a small, monthly changing palette of
older titles.


\subsubsection{\Gls{cg} Ecosystems}

NVIDIA's \gls{cg} platform \gfnow
is available in North America and select European countries.
In Germany the service currently offers $55$ PC titles
for a monthly subscription fee of \SI{10}[\EUR]{}. An additional
per-game one-time fee
is charged for the access to $63$ other games.
The service is delivered from six
specialized data center locations (Dublin and Frankfurt in Europe).

The service requires a rather steep \SI{50}{\mega\bit\per\second} for a full 1080p60 stream.
Streaming is exclusive to \textit{SHIELD} devices.
The \gfnowpc beta program streams to \glspl{PC}, and costs
\$\SIrange{10}{20}{} per hour of usage depending on the desired
visual quality.

\psnow, Sony's cloud gaming service, offers to stream titles from previous
PlayStation generations, as the latest console generation lacks
backwards compatibility.
The service offers 432 games for a monthly flat rate of
\SI{17}[\EUR]{}.
This is in addition to
the device cost: the service is available on PlayStation 4 and 3
consoles and, recently, \glspl{PC}.
Streaming is performed at a resolution of
720p60 requiring a \SI{5}{\mega\bit\per\second}
connection.

\subsection{Utility Metrics}

Utility metrics help to estimate the value of a service for a customer.
It should be noted that no single metric is likely of value by itself,
yet all metrics combined can absolutely influence customer decisions.

The first basic metric is the number of games offered. A larger number
means more choice for the customer, but ``spam'' phenomena need to
be considered, as large offers might include many games of sub-par quality.
Second, the number of players already owning a game can be a
motivating factor for other customers to choose the same game.
One may interpret this as a form of preferential attachment.
This is particularly interesting for decidedly online games, i.e.
such where players interact in the game environment over the network,
so that more interactions are possible if more players are online.

Then, games are priced differently, both within and across
platforms. Many games are offered completely free of charge.
Some are funded indirectly through in-game purchases. On the
other end of the spectrum, titles whose development was expensive
and for which the audience anticipation is high might sell
for a higher price. Furthermore, game lengths come to mind.
Games range from very short (on the order of minutes) to tens, or even
hundreds, of hours (for games with an extensive storyline),
but need not be limited at all (e.g. exploration or ``open world'' games).
Overall, neither mode intrinsically offers higher utility.

Next, the age of games (per their original release date) is of
interest. Some players favor the newest games, others prefer ``classics'', 
which additionally often implies a preference for a specific aesthetic (e.g.
``8-bit'' graphics); any focus or combination thereof will likely attract
specific proportions of customers. Also, newer games tend to also be featured
in the media (and in advertising, increasing their publicity),
offer a greater diversity in game mechanics and control,
and improved technical aspects (resolution, scene complexity).

There also exist considerable amounts of secondary literature about
games in the form of reviews. These can be viewed as social factor,
in that gamers might be drawn to positively-reviewed titles more
strongly, or rather stay away from less favorably reviewed ones.
Some media aggregate reviews from many outlets and
are thus of particular interest for an analysis. Many other utility
metrics are imaginable, including combinations of the metrics mentioned above.

\subsection{Objective Data Sources}
In order to investigate the utility metrics described, data was collected from
multiple sources and merged
into a data base. In the interest of repeatability and
reproducibility, all of the data reported on in this work, as well as
the code used to collect and process it, can be found in public
repositories\footnote{The repository resides at \url{https://github.com/mas-ude/cost-of-cloud-gaming}}.
Data were joined on game names and platforms (where appropriate).
The different data sources are described in the following.
For \steam, the public \acrshort{REST} \acrshort{API} was used to
fetch the name and current price of each game at different points in
time.
This data was combined with \acrshort{API} data from the 3rd-party site
\textit{SteamSpy}\footnote{\url{https://steamspy.com}}, which parses all
publicly visible \steam user profiles and
estimates statistics on the size of the player base and the time each
player spends with a title. \textit{SteamSpy} also provides a heuristic
projection of the total number of owners of each listed title on \steam.
For \gfnow, game names and prices were screen-scraped manually.
Game names for \psnow is available from the PlayStation
website\footnote{\url{https://www.playstation.com/de-de/explore/playstation-now/ps-now-games/}}.
Thus, the platform portfolios can be compared.

Game publishers seldom publish intended playthrough
length of games, and not all games necessarily have one. However, players
may self-report their experienced playthrough times on sites like
\hltb\footnote{\url{http://howlongtobeat.com/}}, which was web-scraped
for this analysis\footnote{\url{https://github.com/mas-ude/gamelengths-scraper}}.
Lastly, the age of a game is computable from its release date. To this end, the
\metacritic\footnote{\url{http://www.metacritic.com/}} page which
aggregates reviews of video games (and other media) was
scraped\footnote{\url{https://github.com/mas-ude/metacritic_scraper}}.



\subsection{Subjective Data}

The subjective data for this paper was collected through an online
survey using the \textsc{SoSci Survey} web tool. The link was shared
in the authors' social networks, and posted in various sub-forums on
the \textsc{Reddit} online community web page.
The survey included $9$ demographic questions, $12$ questions about the
participants' personal gaming habits and history, and $7$ service/system
influence related questions, not all of which are evaluated in full
in this paper.
In order to keep the time to completion reasonable, most questions
were stated either as yes/no questions, or showed predefined answers
to be ranked on a five-level Likert scale, e.g. from ``extremely
important'' to ``not important at all''.
The complete questionnaire and anonymized responses
are reproduced in rhe repository above.
The survey was completed by $488$ participants.

The context influence factors surveyed include social context (such
as the importance of online aspects in games and the influence of
friends and social media on buying decisions), novelty (in the sense
of new, recently published games), and service factors (referring to
properties of online marketplaces with respect to the utility metrics
discussed before.) Furthermore, system influence factors such as
the preferred gaming hardware platform and available input devices
are studied.
