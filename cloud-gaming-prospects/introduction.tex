%!TEX root = paper.tex
%%%%%%%%%%%%%%%%%%%%%%%%%%%%%%%%%%%%%%%%%%%%%%%%%%%%%%%%%%%%%%%%%%%%%%%%
\section{Introduction}

Cloud gaming has become quite a popular research topic in recent years and since the initial publishing in 2016 has been commercially targeted with a second wave of cloud platforms, whose merits are still to be determined. Much of the existing research is aimed at comparing the experienced quality of cloud gaming to that of conventional gaming approaches, often with reasonable results for the streaming approach (cf., for example, \cite{5976180}). So, if there are only negligible quality drawbacks, what about the commercial success of cloud gaming? Intuitively, one would assume that this could yield substantial benefits in terms of cost and flexibility as a result of scaling effects through the used cloud gaming hardware in comparison to equipment-heavy classical home gaming approaches. However, the cloud gaming market seems to stagnate with a high rate of fluctuation on the market, i.e., a constant stream of market entrances and exits. For example, one of the most prominent services in the past, \textsc{OnLive}, ceased to exist in 2015.

Many cloud gaming approaches that vary by means of technical, service and pricing model differences, have been tested so far, but the public interest remains low. This might be attributed to the broad range of available, established substitutes, e.g., non-cloud gaming platforms such as video game consoles or PCs --- with \steam\footnote{\url{http://store.steampowered.com/}} as one of the largest contenders. The move to digital distribution made gaming on PCs quite popular, and PC games pricing became much more dynamic and affordable in the process.

On the surface, current cloud gaming services attempt to adopt a \textit{fixed fee subscription} model over the traditional \textit{à la carte} model. Fixed fee subscription proved to be hugely successful for other types of media, e.g., \textsc{Netflix} for movies and shows or \textsc{Spotify} for music. However, these two types of services offer a much larger catalogue of content at a comparable or even lower price than cloud gaming services. Additionally, streaming asynchronous non-interactive media is technically less demanding (and thus cheaper to operate) than maintaining a quality level on par with that of locally running games.

The two main research questions that this work aims to tackle are thus: \textit{``Can cloud gaming be attractive for users in today's highly competitive market?''} and \textit{``Can you operate a cloud gaming service with acceptable margins while maintaining acceptable quality levels?''}. Both questions are strongly intertwined as in order to make such services attractive one would have to offer sufficient quality and quantity of games with a competitive pricing while not operating at a loss. Given the current market situation, one could actually paraphrase both questions into one: \textit{``Can you compete with the PC gaming and Steam ecosystem (in terms of quality, prices, and variety)?''}

In order to answer these questions, this paper looks at the perspectives of users and service providers separately, and provides arguments backed by data and simple models. To investigate the customer's perspective we employ domain-specific user engagement metrics (like review scores and lengths and playtimes of games) to compare various services, cloud gaming as well as conventional, to each other. Additionally, using this data models are set up that project the value (in terms of the number of games) a user gets for a certain amount of money. We find that in the investigated cases the cloud gaming services' offer is limited, yet still charges relatively high prices, thus reducing the attractiveness for users in comparison to alternative services. We find this to apply especially for gamers with higher budgets and interest in a larger selection of games.

Due to the limited amount of freely available data on operating a cloud gaming service, the perspective of the cloud gaming operator is investigated by setting up efficiency models centered around the analysis of overbooking practices for server resources. Our initial results hint at the problematic nature of cloud gaming in terms of scaling and cost efficiency. When compared to other cloud services that achieve high values of cost efficiency and capacity utilization, we believe that cloud gaming platforms will be much more peak-oriented and thus achieve much lower values of server utilization. The end-to-end lag requirements of games demand servers located in the user's vicinity, which eliminates most multiplexing gains that a centralized data center could garner over the course of a day. Thus, the scaling benefits may be substantially lower than for other cloud services. Additionally, games require dedicated hardware support, which is of less use to most other cloud service use cases, diminishing the potential of cross-service reuse.

These initial insights do not shed a bright light on the commercial future of cloud gaming services in general. Unless major cost reductions are achieved, while the streaming quality is maintained or even improved, the future of cloud gaming might be bleak. But there still might be some niches to place a cloud gaming service where the competition is less strong --- a route that one of the current cloud gaming services already takes. We plan to take a deeper look at all these aspects and provide more detailed models in the future.

~\\
This paper is structured as follows: §~\ref{sec:relatedwork} provides a brief overview of the related work. Afterwards, §~\ref{sec:background} explains the necessary terms and technical details. The main part of this work encompass §§~\ref{sec:engagement} and \ref{sec:suppliermodelling}, which conduct the dual-perspective investigation of the cloud gaming providers' service offering and business case. The paper concludes in §~\ref{sec:conclusion} with some remarks and an outlook.
