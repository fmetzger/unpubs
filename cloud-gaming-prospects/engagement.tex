%!TEX root = paper.tex
%%%%%%%%%%%%%%%%%%%%%%%%%%%%%%%%%%%%%%%%%%%%%%%%%%%%%%%%%%%%%%%%%%%%%%%%%%%%%%%
\section{User Perspective and Prospects}
\label{sec:engagement}

This section interprets the previous results from the point of view of an end customer. Section~\ref{subsec:metrics} reviews game and platform characteristics as possible engagement metrics, i.e., measures for the attractiveness of a platform. Section~\ref{subsec:cost-benefit-models} constructs two simple cost/benefit models based on the platform costs and the number of games affordable as a benefit metric.


%%%%%%%%%%%%%%%%%%%%%%%%%%%%%%%%%%%%%%%%%%%%%%%%%%%%%%%%%%%%%%%%%%%%%%%%%%%%%%%%
\subsection{Engagement Metrics}\label{subsec:metrics}

The benefit that platform and game characteristics create for different types of gamers are diverse, which will be reflected in the following interpretations.

\subsubsection{Number of games} The number of games for the three platforms are widely dissimilar, making the choice unanimous for gamers that favour wider ranges of offers. Other factors motivating different platform choices might be found in already-owned console hardware, or platform-exclusive titles.

\subsubsection{Age of games} This has multiple possible effects on attractiveness. Older games might be more likely to be considered ``classics''. On the other hand, newer games tend to also be topics in the media (and in advertising, increasing their publicity), offer a greater diversity (e.g., regarding game mechanics and control), % Wii once was new!
and feature improved technical quality (resolution, scene complexity).

\subsubsection{Game lengths} It can be argued that engagement due to game length should be considered differently depending on the player type. For casual gamers, shorter games probably have more appeal, as their limited playtime is more likely to allow players to complete them. ``Hardcore'' gamers that spend more time gaming overall might find longer games more engaging. If a gaming platform wants to attract members of both crowds, it should thus offer a wide range of game lengths. Figure~\ref{fig:rel-combinedlength-owners} plots the number of owners of games on \steam over the combined game length from the \hltb dataset. A small positive effect exists (Pearson correlation $r = 0.175$), indicating that longer games attract more buyers, or (swapping cause and effect) that more people tend to buy longer games.

\subsubsection{Review scores} Published scores for games constitute an additional social factor, in that gamers might be drawn to positively-reviewed titles more strongly. Taking game ownership as a possible indicator for the popularity of a game, the \metacritic dataset exhibits a small positive correlation between general score and number of game ownerships on \steam (Pearson correlation coefficient $r = 0.199$).

\begin{figure}[!t]
	\centering
	\includegraphics[width=1.0\columnwidth]{images/rel-combinedlength-owners.pdf}
	\caption{Relationship of game lengths to ownership.}
\label{fig:rel-combinedlength-owners}
\end{figure}


%%%%%%%%%%%%
\subsubsection{Further Potential Engagement Factors}

Due to the limited amount of available data the number of currently observable potential engagement metrics is restricted. However, many more come to mind and are worth investigating in the future. These could include:

\begin{itemize}
	\item the number of platform ``exclusive'' game titles,
	\item the genre as well as other classifications of games,
	\item the number of game sales and subscriber numbers,
	\item technical aspects like graphical fidelity, performance, precision and responsiveness of controls,
	\item measures of the game's content like variety and quality of game mechanics,
	\item or other content-centric factors.
\end{itemize}


%%%%%%%%%%%%%%%%%%%%%%%%%%%%%%%%%%%%%%%%%%%%%%%%%%%%%%%%%%%%%%%%%%%%%%%%%%%%%%%%
\subsection{Cost-Benefit Models}\label{subsec:cost-benefit-models}

This section now turns to modeling gamer benefit as a function of cost: The number of games playable, given the amount of money paid. The two models presented express how many games a given budget will buy on the different platforms, as a function of the customer's budget, and over years, given a constant yearly expenditure. (Other benefits could be modeled as well, e.g., the distribution of review scores or hours of gameplay, but this is left for future work.)

A few basic assumptions are made. The initial hardware costs are factored in proportionately to the expected life and depreciation time of the devices, which is assumed to be \SI{3}{\year} for PCs, and \SI{6}{\year} for the consoles and devices necessary to receive the streaming service. 
The difference in the service models between the ownership model of \steam, the hybrid subscription plus permanent rental model of \gfnow, as well as the subscription and timed rental model of \psnow were discussed in §~\ref{subsec:platform-characteristics} already. Since it is difficult to express the different notions of ``being able to play'' and ``owning'' a game in simple terms, our benefit models just consider the number of games that one has access to and that could have been played at one point. (This also ignores other means of acquiring games cheaper from alternative storefronts).
Lastly, the numbers of games available per platform and the game and subscription fees are considered as fixed at their current values.


%%%%%%%%%%%%%%%%%%%%%%%%%%%%%%%%%%%%%%%%%%%%%%
\paragraph{Affordable Games on a Budget Model}

The first model assumes that one has a fixed budget to spend on video games. This model then shows, depending on the size of the budget, how many games one would be able to afford considering all initial and continual costs.
Figure~\ref{fig:gamesperyear-over-budget} shows that all platforms start with relatively high fixed costs (including hardware depreciation and subscription fees as applicable). The subscription models provide instant access to a certain amount of titles (visible as ``steps upwards'' in the available game count) which could make them more attractive to newcomers on a low budget. However, both cloud platforms exhibit plateaus due to the relatively small selection of games available. On the other hand, \steam's offer of more than $7000$ titles is far from depleted at above \SI{1000}[\EUR]{} worth of games.

\begin{figure}[!t]
	\centering
	\includegraphics[width=1.0\columnwidth]{images/gamesperyear-over-budget.pdf}
	\caption{Affordable games per year on different platforms, given the customer's budget}
\label{fig:gamesperyear-over-budget}
\end{figure}


%%%%%%%%%%%%%%%%%%%%%%%%%%%%%%%%%%%%%%%%%%%
\paragraph{Affordable Games per Year Model}

The second model extends the previous one to span a period of multiple years, investigating the value one gets if a constant annual budget is spent. For the exemplary model, the budget is assumed to be \SI{500}[\EUR]{}.
Figure~\ref{fig:games-over-years} depicts this example over ten years. On the two cloud platforms, the continual subscription costs limit the remaining budget for additional rental titles until the maximum number of titles is reached with that particular service, whereas the number of titles from \steam climbs steadily. The benefits of a multi-year commitment to these cloud gaming services therefore seem to be very limited, especially when considering that no games are retained after ending the subscription.

Increasing the annual budget further (to \SI{1000}[\EUR]{}) exhausts the limited number of games on \psnow and \gfnow in or after the first year, respectively. Conversely, if only \SI{260}[\EUR]{} per year are available to spend on gaming, the cloud platforms' offers are larger than \steam's for four to six years.

\begin{figure}[!t]
	\centering
	\includegraphics[width=1.0\columnwidth]{images/games-over-year.pdf}
	\caption{Number of affordable games per platform, assuming a yearly customer expenditure of \SI{500}[\EUR]{}.}
\label{fig:games-over-years}
\end{figure}


%%%%%%%%%%%%%%%%%%%%%%%%%%%%%%%%%%%%%%%%%%%%%%%%%%%%%%%%%%%%%%%%%%%%%%%%%%%%%%%%
\subsection{Discussion}

This section mapped game and platform characteristics into simple user engagement metrics. For example, it is plausible that platforms with more, newer, and better-rated games attract more gamers; diversely distributed game lengths could be seen as catering to casual and dedicated gamer crowds likewise.
In terms of cost considerations, our models indicate that the current offers of \gfnow and \psnow best fit gamers on low budgets: Limitations on the amount of money that can be spent may somewhat mask the smaller range of offered games. If this is true, it further increases the economical pressure on cloud gaming platforms (as discussed in the following section). Overall, these investigations  expose the difference between an open market platform and the curated-by-necessity cloud gaming services.
