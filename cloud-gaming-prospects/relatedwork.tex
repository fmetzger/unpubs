%!TEX root = paper.tex
%%%%%%%%%%%%%%%%%%%%%%%%%%%%%%%%%%%%%%%%%%%%%%%%%%%%%%%%%%%%%%%%%%%%%%%%%%%%%%%
\section{Related Work}
\label{sec:relatedwork}

This section first revisits operational benefits of cloud gaming and cloud services in general, but also encompasses energy consumption issues. Furthermore, literature on \gls{QoE} aspects of gaming in general and cloud gaming in particular is covered. These reflect the users' quality expectations but can also outline the requirement for a service's operation.


%%%%%%%%%%%%%%%%%%%%%%%%%%%%%%%%%%%%%%%%%%%%%%%%%%%%%%%%%%%%%%%%%%%%%%%%%%%%%%%
\subsection{Operational and Efficiency Factors}

Many publications on cloud gaming only consider the client's side and often restrict their view to mobile cloud gaming. For example, \cite{Soliman2013} overviews some general issues of mobile cloud gaming. Several other publications, e.g., \cite{6924295} and \cite{Huang:2014:MCP:2755535.2755542} investigate the client device's energy saving potential of mobile cloud gaming but find rather marginal energy savings. A 2014 publication \cite{6882299} describes some potential benefits of a centralized cloud gaming platform, however operates under questionable assumptions. Two further papers (\cite{6853364} and \cite{6365107}) suggest an optimization model to place and provision cloud gaming \acrshortpl{VM} in order for a service provider to operate at profits. The impact on \gls{QoE}, however, seems to be significant but is concealed through the lack of absolute data given. The efficient placement and selection of servers is focal for Cloud services (e.g., \cite{6740249}), but the optimization potential is limited for cloud gaming due to more stringent requirements both in terms of compute resources as well as latency. This restricts the server selection problem to relatively narrow geographic regions.


%%%%%%%%%%%%%%%%%%%%%%%%%%%%%%%%%%%%%%%%%%%%%%%%%%%%%%%%%%%%%%%%%%%%%%%%%%%%%%%
\subsection{Gaming QoS/QoE}

The quality of games is revisited along the axes of \gls{E2E} lag, image quality, and frame rates.
Considerable research efforts have been put into the network delay component of the \gls{E2E} lag for both online games and cloud gaming. The results, however, remain inconclusive. Studies of multiplayer games often focused on \glspl{FPS} such as \textit{Quake 3}~\cite{1266180} or \textit{Unreal Tournament 2003}~\cite{Beigbeder:2004:ELL:1016540.1016556}. Concerning cloud gaming, Chen et al.~\cite{6670099} for example find very high and variable delay values even when neglecting the network delay. Furthermore, \cite{Choy:2012:BSC:2501560.2501563} gives some insights on the delay requirements of streamed games and the implications for data center distance as well as placement.

Image quality represents a further \gls{QoE} factor. Gaming adds another dimension to typical image quality assessments, as most games allow for changes to their graphical fidelity, be it either the resolution or more demanding graphical features, such as ambient occlusion or anti-aliasing. Cloud gaming usually locks these options at one specific setting for a specific quality-to-resource-demand trade-off, resulting in an often lower source quality than what local games can offer. As an example, the work in \cite{slivarimpact} takes a look at different encoding parameters for cloud gaming.

Finally, and often neglected, is the game's frame rate and the streaming frame rate \cite{metzger2016gamesframes,Metzger2016}. Due to the interactivity of the media the requirements are generally higher than for video streaming, e.g., \SI{60}{\hertz} is an accepted standard for many games. Too low frame rates will result in a reduced quality due to observable stuttering and issues with inputting commands.
An overview of some further \gls{QoE} taxonomy and influence factors especially for mobile games is given in \cite{beyer2014typedisplaydelayimpact}. Several efforts also set up subjective tests of cloud gaming services with specific \gls{QoS} parameters in mind. Such studies can be found in, e.g., \cite{Jarschel20132883} and  \cite{6614351}. Efforts have also been made towards an \acrshort{ITU-T} recommendation for subjective game testing as reported in \cite{mollertowards}.

