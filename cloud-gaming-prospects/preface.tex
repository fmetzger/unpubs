%!TEX root = paper.tex
%%%%%%%%%%%%%%%%%%%%%%%%%%%%%%%%%%%%%%%%%%%%%%%%%%%%%%%%%%%%%%%%%%%%%%%%%%%%%%%%
\section*{Preface}


\textit{This paper was originally written in 2016 as a collaboration between colleagues that are or have previously been at the University of Duisburg-Essen, the University of Vienna, and the Austrian Institute of Technology (AIT), but has never never been published. Since then the second wave of cloud gaming platforms have emerged and the paper has, subsequently, been updated in 2021.}

\textit{Even today we think it is important to understand why the first wave of commercial cloud gaming services failed to take off in order to understand the merit of the current second wave of services. 
While this paper definitely did not provide conclusive answers to this question, it does offer some unique perspectives and discussions from a user- and operator-centric techno-economic point of view.
We think, these ideas might be of even more merit today, and might help to better assess the current second wave of cloud services. We are also still engaged in this line of research, and we strive to continue this work in the future, and are looking for scientific exchange on cloud gaming platforms.} 

\textit{
Therefore, we wanted to preserve this work in the form it was originally prepared in 2016, with updated affiliations, restored links, other minor updates and corrections, and this preface. Past reviewers of this manuscript mostly criticized a bias against new cloud services, given that they were comparatively young and, thus, naturally had a more limited offering. In addition it was criticized that the underlying quality issues, strict requirements, and additionally incurred costs to the users (e.g., more expensive data plans) of cloud gaming were not tackled here. Our discussion points were based on very specific subjective data (e.g., review scores, game lengths), which might not be entirely representative of users and platforms.
On the other hand, reviewers praised the interesting and novel discussions, based on real data and in comparison of multiple platforms using pricing models and engagement metrics.}
