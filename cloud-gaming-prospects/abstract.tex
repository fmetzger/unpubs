%!TEX root = paper.tex
%%%%%%%%%%%%%%%%%%%%%%%%%%%%%%%%%%%%%%%%%%%%%%%%%%%%%%%%%%%%%%%%%%%%%%%%%%%%%%%%
\begin{abstract}

In recent years, cloud gaming has become a popular research topic and has claimed many benefits in the commercial domain over conventional gaming. While, cloud gaming platforms have frequently failed in the past, they have received a new impetus over the last years that brought it to the edge of commercial breakthrough. The fragility of the cloud gaming market may be caused by the high investment costs, offered pricing models or competition from existing ``à la carte'' platforms. This paper aims at investigating the costs and benefits of both platform types through a twofold approach. We first take on the perspective of the customers, and investigate several cloud gaming platforms and their pricing models in comparison to the costs of other gaming platforms. Then, we explore engagement metrics in order to assess the enjoyment of playing the offered games. Lastly, coming from the perspective of the service providers, we aim to identify challenges in cost-effectively operating a large-scale cloud gaming service while maintaining high \acrshort{QoE} values. Our analysis provides initial, yet still comprehensive reasons and models for the prospects of cloud gaming in a highly competitive market.

\end{abstract}
