%!TEX root = paper.tex
%%%%%%%%%%%%%%%%%%%%%%%%%%%%%%%%%%%%%%%%%%%%%%%%%%%%%%%%%%%%%%%%%%%%%%%%%%%%%%%%
\begin{abstract}

In recent years, cloud gaming has become a popular research topic, and the list of its supposed benefits is long. However, cloud gaming platforms are still waiting for the commercial breakthrough. This might be caused by the pricing models and product offerings by existing ``à la carte'' platforms. This paper aims at investigating the costs and benefits of both platform types through a twofold approach. We first take on the perspective of the customers, and investigate several cloud gaming platforms and their pricing models in comparison to the costs of ``à la carte'' platforms. Then, we explore engagement metrics in order to assess the enjoyment of playing the offered games. Lastly, coming from the perspective of the service providers, we aim to identify challenges in cost-effectively operating a large-scale cloud gaming service while maintaining high \acrshort{QoE} values. Our analysis provides initial, yet still comprehensive reasons and models for the prospects of cloud gaming in a highly competitive market.

\end{abstract}
