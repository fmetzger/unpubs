%!TEX root = paper.tex
%%%%%%%%%%%%%%%%%%%%%%%%%%%%%%%%%%%%%%%%%%%%%%%%%%%%%%%%%%%%%%%%%%%%%%%%%%%%%%%%
\section{Conclusion}
\label{sec:conclusion}

The approach taken here attempts to combine the strengths of participatory crowdsensing with the knowledge garnered by the \gls{QoE} community in recent years in order to provide the public with a type of information --- the attainable YouTube video quality at a specific location in a mobile network--- that has much more immediate utility than providing simple measured network \gls{QoS} samples. Such a large-scale crowdsensing endeavor can only be achieved through some compromises as they have been taken here. Actively conducting subjective video \gls{QoE} user studies at each location is equally unfeasible as is being wasteful with the participants' devices' resources, especially with stringent bandwidth data caps in place.

All in all, we think that our approach of conducting large scale low-volume bandwidth estimation measurements and transforming them through adequate \gls{QoE} mapping might be much more engaging for a wider audience and thus indicative of the future to come.