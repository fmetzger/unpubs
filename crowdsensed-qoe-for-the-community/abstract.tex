%!TEX root = paper.tex
%%%%%%%%%%%%%%%%%%%%%%%%%%%%%%%%%%%%%%%%%%%%%%%%%%%%%%%%%%%%%%%%%%%%%%%%%%%%%%%%
\begin{abstract}

In recent years several community testbeds as well as participatory sensing  platforms have successfully established themselves to provide open data to everyone interested. Each of them with a specific goal in mind, ranging from collecting radio coverage data up to environmental and radiation data. Such data can be used by the community in their decision making, whether to subscribe to a specific mobile phone service that provides good coverage in an area or in finding a sunny and warm region for the summer holidays.

However, the existing platforms are usually limiting themselves to directly measurable network \acrshort{QoS}. If such a crowdsourced data set would provide more in-depth derived measures, this would enable an even better decision making. A community-driven crowdsensing platform that derives spatial application-layer user experience from resource-friendly bandwidth estimates would be such a case, video streaming services come to mind as a prime example. In this paper we present a concept for such a system based on an initial prototype that eases the collection of data necessary to determine mobile-specific \acrshort{QoE} at large scale. In addition we reason why the simple quality metric proposed here can hold its own.
\end{abstract}
